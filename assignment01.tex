\documentclass{article}
\usepackage{kotex}
\begin{document}
git은 사용자가 공유된 파일을 편집할때 유용한 버전 관리 시스템이다. 사용자가 파일 및 브랜치를 추가하거나 변경할때마다, 해당 변경에 대한 commit이 버전별로 git에 저장되어, 수정자 뿐만 아니라 파일을 공유하는 사용자들도 commit 기록을 확인하여 변경 이력과 내용을 알 수 있다.  
\\git은 사용자의 컴퓨터의 로컬 저장소와는 다른 저장소인 원격 저장소를 제공하여주는데, 이 저장소는 clone, pull, push 명령어를 이용하여 사용할 수 있다. Clone은 원격 저장소에 있는 데이터를 통째로 복사하여 로컬 저장소에 저장하는 것이다. Pull은 다른 사용자가 push한 변경 내용을 자신의 로컬 저장소에도 적용하는 것이다. Push는 자신의 로컬 저장소에서 작업한 변경 사항을 원격 저장소로 옮겨 공유하는 것이다. 별로도 merge가 있는데 merge는 자신이 pull을 실행하고 작업하는 동안 다른 사용자가 push를 하여 자신이 pull했던 저장소가 최신 버전이 아닐 경우 자신의 push 요청이 거부되는 것을 막기 위하여 다른 사용자의 업데이트 이력을 내 저장소에도 갱신하는 것이다.  
\\추가로 브랜치라는 시스템이 있다. 브랜치는 독집적인 시스템으로서 브랜치가 다른 브랜치에 영향을 끼치지 않기 때문에, 동시에 여러 버전의 작업을 진행 혹은 저장할 수 있게 해준다. 원격 저장소를 새로 만들때 기본적으로 master브랜치가 생성되는데, 추가적으로 브랜치를 만들지 않으면 작업들이 master 브랜치에서 일어난다.
\end{document}